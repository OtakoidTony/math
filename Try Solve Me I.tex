\documentclass[fleqn]{article}
\usepackage[left=1in, right=1in, top=1in, bottom=1in]{geometry}
\usepackage{mathexam}
\usepackage{amsmath}

\let\ds\displaystyle
\usepackage[cjk]{kotex}

\ExamClass{High School Mathematics}
\ExamName{Try solve this!}
\ExamHead{\today}
\begin{document}

\ExamNameLine
\ExamInstrBox{
\textbf{제시문 A}\\\\
 함수 $S(x)$를 다음과 같이 정의한다.
\[
S(x)=\sum_{n=0}^\infty x^n
\]

\textbf{제시문 B}\\\\
 복소수$z$가 $x+yi$일 때 복소수$z$의 크기 $|z|$를 $\sqrt{ x^2 + y^2 }$으로 정의한다.\\
 
\textbf{제시문 C}\\\\
 제시문 A에서의 함수$S(x)$에 대하여 함수 $f(x)$를 다음과 같이 정의한다.
 \[
 f(x)=\int_0^x{S(t)S(-t)dt}
 \]
}

\begin{enumerate}
   \item 복소수 $z$에 대하여 $z=x+yi$라 하자. 이 때 $S(z)$가 수렴하게 하는 $z$값들의 집합을 $(x,y)$에 관한 조건제시법으로 정의된 집합으로 표현하여라. (예를 들어 $3+4i$는 $(3, 4)$이다.)
        \noanswer[1in]
   \item $S(z)$가 수렴하게 하는 $z$값들의 집합을 $C$라 하자. 이 때, C의 넓이를 구하여라. 
        \noanswer[1in]
   \item $f(x)$의 값이 실수인 $x$의 최댓값과 최소값을 각각 $a,b$라 하자. 이 때 $a,b$의 값을 각각 구하고 $\int_b^a{f(t)}dt$를 구하여라.
        \noanswer[1in]        

\end{enumerate}
\end{document}

